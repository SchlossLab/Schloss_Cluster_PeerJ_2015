\documentclass[11pt,]{article}
\usepackage{lmodern}
\usepackage{amssymb,amsmath}
\usepackage{ifxetex,ifluatex}
\usepackage{fixltx2e} % provides \textsubscript
\ifnum 0\ifxetex 1\fi\ifluatex 1\fi=0 % if pdftex
  \usepackage[T1]{fontenc}
  \usepackage[utf8]{inputenc}
\else % if luatex or xelatex
  \ifxetex
    \usepackage{mathspec}
    \usepackage{xltxtra,xunicode}
  \else
    \usepackage{fontspec}
  \fi
  \defaultfontfeatures{Mapping=tex-text,Scale=MatchLowercase}
  \newcommand{\euro}{€}
\fi
% use upquote if available, for straight quotes in verbatim environments
\IfFileExists{upquote.sty}{\usepackage{upquote}}{}
% use microtype if available
\IfFileExists{microtype.sty}{%
\usepackage{microtype}
\UseMicrotypeSet[protrusion]{basicmath} % disable protrusion for tt fonts
}{}
\usepackage[margin=1.0in]{geometry}
\ifxetex
  \usepackage[setpagesize=false, % page size defined by xetex
              unicode=false, % unicode breaks when used with xetex
              xetex]{hyperref}
\else
  \usepackage[unicode=true]{hyperref}
\fi
\hypersetup{breaklinks=true,
            bookmarks=true,
            pdfauthor={},
            pdftitle={Application of database-independent approach to assess the quality of OTU picking methods},
            colorlinks=true,
            citecolor=blue,
            urlcolor=blue,
            linkcolor=magenta,
            pdfborder={0 0 0}}
\urlstyle{same}  % don't use monospace font for urls
\usepackage{graphicx,grffile}
\makeatletter
\def\maxwidth{\ifdim\Gin@nat@width>\linewidth\linewidth\else\Gin@nat@width\fi}
\def\maxheight{\ifdim\Gin@nat@height>\textheight\textheight\else\Gin@nat@height\fi}
\makeatother
% Scale images if necessary, so that they will not overflow the page
% margins by default, and it is still possible to overwrite the defaults
% using explicit options in \includegraphics[width, height, ...]{}
\setkeys{Gin}{width=\maxwidth,height=\maxheight,keepaspectratio}
\setlength{\parindent}{0pt}
\setlength{\parskip}{6pt plus 2pt minus 1pt}
\setlength{\emergencystretch}{3em}  % prevent overfull lines
\providecommand{\tightlist}{%
  \setlength{\itemsep}{0pt}\setlength{\parskip}{0pt}}
\setcounter{secnumdepth}{0}

%%% Use protect on footnotes to avoid problems with footnotes in titles
\let\rmarkdownfootnote\footnote%
\def\footnote{\protect\rmarkdownfootnote}

%%% Change title format to be more compact
\usepackage{titling}

% Create subtitle command for use in maketitle
\newcommand{\subtitle}[1]{
  \posttitle{
    \begin{center}\large#1\end{center}
    }
}

\setlength{\droptitle}{-2em}
  \title{\textbf{Application of database-independent approach to assess the
quality of OTU picking methods}}
  \pretitle{\vspace{\droptitle}\centering\huge}
  \posttitle{\par}
  \author{}
  \preauthor{}\postauthor{}
  \date{}
  \predate{}\postdate{}

\usepackage{helvet} % Helvetica font
\renewcommand*\familydefault{\sfdefault} % Use the sans serif version of the font
\usepackage[T1]{fontenc}

\usepackage[none]{hyphenat}

\usepackage{setspace}
\doublespacing
\setlength{\parskip}{1em}

\usepackage{lineno}

\usepackage{pdfpages}
\usepackage{comment}

% Redefines (sub)paragraphs to behave more like sections
\ifx\paragraph\undefined\else
\let\oldparagraph\paragraph
\renewcommand{\paragraph}[1]{\oldparagraph{#1}\mbox{}}
\fi
\ifx\subparagraph\undefined\else
\let\oldsubparagraph\subparagraph
\renewcommand{\subparagraph}[1]{\oldsubparagraph{#1}\mbox{}}
\fi

\begin{document}
\maketitle

\begin{center}
\vspace{25mm}
Patrick D. Schloss${^\dagger}$

\vspace{30mm}

$\dagger$ To whom correspondence should be addressed: pschloss@umich.edu

Department of Microbiology and Immunology, University of Michigan, Ann Arbor, MI
\end{center}

\newpage

\linenumbers

\subsection{Abstract}\label{abstract}

Assigning 16S rRNA gene sequences to operational taxonomic units (OTUs)
allows microbial ecologists to overcome the inconsistencies and biases
within bacterial taxonomy and provides a strategy for clustering similar
sequences that do not have representatives in a reference database. I
have applied the Matthew's correlation coefficient to assess the ability
of 15 reference-independent and -dependent clustering algorithms to
assign sequences to OTUs. This metric quantifies the ability of an
algorithm to reflect the relationships between sequences without the use
of a reference and can be applied to any dataset or method. The most
consistently robust method was the average neighbor algorithm; however,
for some datasets other algorithms matched its performance.

\newpage

Numerous algorithms have been developed for solving the seemingly simple
problem of assigning 16S rRNA gene sequences to operational taxonomic
units (OTUs). These algorithms were recently the subject of benchmarking
studies performed by Westcott and myself (1, 2), He et al (3), and
Kopylova et al (4). These studies provide a thorough review of the
sequencing clustering landscape, which can be divided into three general
approaches: (i) \emph{de novo} clustering where sequences are clustered
without first mapping sequences to a reference database, (ii)
closed-reference clustering where sequences are clustered based on the
references that the sequences map to, and (iii) open reference
clustering where sequences that do not map adequately to the reference
are then clustered using a \emph{de novo} approach. Assessing the
quality of the clustering assignments has been a persistent problem in
the development of clustering algorithms.

The recent analysis of Kopylova et al (4) repeated many of the
benchmarking strategies employed by previous researchers. Many algorithm
developers have clustered sequences from simulated communities or
sequencing data from synthetic communities of cultured organisms and
quantified how well the OTU assignments matched the organisms' taxonomy
(5--16). Although an OTU definition would ideally match bacterial
taxonomy, bacterial taxonomy has proven itself to be fluid and reflect
the biases of various research interests. Furthermore, it is unclear how
the methods scale to sequences from the novel organisms we are likely to
encounter in deep sequencing surveys. In a second approach, developers
have compared the time and memory required to cluster sequences in a
dataset (6, 13, 17, 18). These are valid parameters to assess when
judging a clustering method, but indicate little regarding the quality
of the OTU assignments. For example, reference-based methods are very
efficient, but do a poor job of reflecting the genetic diversity within
the community when novel sequences are encountered (2). In a third
approach, developers have compared the number of OTUs generated by
various methods for a common dataset (4, 5). Although methods need to
guard against excessive splitting of sequences across OTUs, by focusing
on minimizing the number of OTUs in a community developers risk
excessively lumping sequences together that are not similar. In a fourth
approach, a metric of OTU stability has been proposed as a way to assess
algorithms (3). Although it is important that the methods generate
reproducible OTU assignments when the initial order of the sequences is
randomized, this metric ignores the possibility that the variation in
assignments may be equally robust or that the assignments by a highly
reproducible algorithm may be quite poor. In a final approach, some
developers have assessed the quality of clustering based on the method's
ability to generate the same OTUs generated by other methods (18, 19).
Unfortunately, without the ability to ground truth any method, such
comparisons are tenuous. There is a need for an objective metric to
assess the quality of OTU assignments.

Westcott and I have proposed an unbiased and objective method for
assessing the quality of OTU assignments that can be applied to any
collection of sequences (1, 2). Our approach uses the observed
dissimilarity between pairs of sequences and information about whether
sequences were clustered together to quantify how well similar sequences
are clustered together and dissimilar sequences are clustered apart. To
quantify the correlation between the observed and expected OTU
assignments, we synthesize the relationship between OTU assignments and
the distances between sequences using the Matthew's correlation
coefficient (20). I have expanded our previous analysis to evaluate
three hierarchical and seven greedy \emph{de novo} algorithms, one
open-reference clustering algorithm, and four closed-reference
algorithms (Figure 1). To test these approaches I applied each of them
to datasets from soil (21), mouse feces (22), and two simulated
datasets. The simulated communities were generated by randomly selecting
10,000 16S rRNA sequences that were unique within the V4 region from the
SILVA non-redundant database (4, 23). Next, an even community was
generated by specifying that each sequence had a frequency of 100 reads
and a staggered community was generated by specifying that the abundance
of each sequence was a randomly drawn a uniform distribution between 1
and 200. A reproducible version of this manuscript and analysis has been
added to the repository available at
\url{https://github.com/SchlossLab/Schloss_Cluster_PeerJ_2015}.

I replicated the benchmarking approach that I have used previously to
assess the ability of an algorithm to correctly group sequences that are
similar to each other and split sequences that are dissimilar to each
other using the MCC (1, 2). When I compared the MCC values calculated
using the ten \emph{de novo} algorithms with the four datasets, the
average neighbor algorithm reliably performed as well or better than the
other methods (Figure 1). For the murine dataset, the MCC values for the
VSEARCH (AGC: 0.76 and DGC: 0.78) and USEARCH-based (AGC: 0.76 and DGC:
0.77) algorithms, Sumaclust (0.76), and average neighbor (0.76) were
similarly high. For each of the other datasets, the MCC value for the
average neighbor algorithm was at least 5\% higher than the next best
method. Swarm does not use a traditional distance-based criteria to
cluster sequences into OTUs and instead looks for natural subnetworks in
the data. When I used the distance threshold that gave the best MCC
value for the Swarm data, the MCC values were generally not as high as
they were using the average neighbor algorithm. The one exception was
for the soil dataset. Among the reference-based methods, all of the MCC
values suffer because when sequences that are at least 97\% similar to a
reference are pooled, the sequences within an OTU could be as much as
6\% different from each other. The effect of this is observed in the MCC
values that were calculated for the OTUs assigned by these methods
generally being lower than those observed using the \emph{de novo}
approaches (Figure 1). It is also important to note that the MCC values
for the closed-reference OTUs are inflated because sequences were
removed from the analysis if there was not a reference sequence that was
more than 97\% similar to the sequence. By choosing to focus on the
ability to regenerate taxonomic clusterings, minimizing the number of
OTUs, and performance Kopylova et al (4) concluded that Swarm and
SUMACLUST had the most consistent performance among the \emph{de novo}
methods. My objective MCC-based approach found that Sumaclust performed
well, but was matched or outperformed by the average neighbor algorithm;
using a 3\% threshold, Swarm was actually one of the worst methods.
Given the consistent quality of the clusterings formed by the average
neighbor algorithm, these results confirm the conclusion from the
previous analysis that researchers should use the average neighbor
algorithm or calculate MCC values for several methods and use the
clustering that gives the best MCC value (2).

Next, I investigated the ability of the reference-based methods to
properly assign sequences to OTUs. The full-length 16S rRNA gene
sequences in the default reference taxonomy that accompanies QIIME are
less than 97\% similar to each other. Within the V4 region, however,
many of the sequences were more similar to each other and even identical
to each other. As a result, we previously found that there was a
dependence between the ordering of sequences in the reference database
and the OTU assignments with USEARCH and VSEARCH (2). To explore this
further, we analyzed the 32,106 unique sequences from the murine dataset
with randomized databases. VSEARCH always found matches for 27,737
murine sequences, the reference matched to those sequences differed
between randomizations. For USEARCH there were between 28,007 and 28,111
matches depending on the order of the reference. In the updated analysis
we found that SortMeRNA resulted in between 23,912 and 28,464 matches.
Using NINJA-OPS with different orderings of the reference sequences
generated the same 28,499 matches. These results point to an additional
problem with closed-reference clustering, which is the inability for the
method to assign sequences to OTUs when a similar reference sequence
does not exist in the database. For the well-characterized murine
microbiota, NINJA-OPS did the best by finding relatives for 88.8\% of
the unique murine sequences. As indicated by the variation in the number
of sequences that matched a reference sequence, these methods varied in
their sensitivity and specificity to find the best reference sequence.
Of the closed-reference methods, NINJA-OPS had the best sensitivity
(99.7\%) and specificity (79.7\%) while SortMeRNA had the worst
sensitivity (95.7\%) and VSEARCH had the worst specificity (60.3\%).
Reference-based clustering algorithms are much faster than \emph{de
novo} approaches, but do not generate OTUs that are as robust.

Although the goal of Kopylova et al (4) was to compare various
clustering algorithms, they also studied these algorithms in the broader
context of raw sequence processing, screening for chimeras, and removal
of singletons. Each of these are critical decisions in a comprehensive
pipeline. By including these steps, they confounded their analysis of
how best to cluster sequences into OTUs. The effect of differences in
MCC values on one's ability to draw inferences is unclear and admittedly
may be relatively minor for some datasets. Because of this uncertainty,
researchers should use the most reliable methods available in case the
differences in clustering do effect the conclusions that can be drawn
from a particular dataset. Through the use of objective criteria that
measure the quality of the clusterings, independent of taxonomy or
database, researchers will be able to evaluate which clustering
algorithm is the best fit for their data.

\newpage

\textbf{Figure 1. Comparison of OTU quality generated by multiple
algorithms applied to four datasets.} The nearest, average, and furthest
neighbor clustering algorithms were used as implemented in mothur
(v.1.37)(24). Abundance (AGC) and Distance-based greedy clustering (DGC)
were implemented using USEARCH (v.6.1) and VSEARCH (v.1.5.0)(3, 5, 25).
Other \emph{de novo} clustering algorithms included Swarm (v.2.1.1)(6,
7), OTUCLUST (v.0.1)(26), and Sumaclust (v.1.0.20). The MCC values for
Swarm were determined by selecting the distance threshold that generated
the maximum MCC value for each dataset. The USEARCH and SortMeRNA
(v.2.0) closed-reference clusterings were performed using QIIME
(v.1.9.1) (27, 28). Closed-reference clustering was also performed using
VSEARCH (v.1.5.0) and NINJA-OPS (v.1.5.0) (16). The order of the
sequences in each dataset was randomized thirty times and the
intra-method range in MCC values was smaller than the plotting symbol.
MCC values were calculated using mothur.

\newpage

\subsection*{References}\label{references}
\addcontentsline{toc}{subsection}{References}

\hypertarget{refs}{}
\hypertarget{ref-Schloss2011Assessing}{}
1. \textbf{Schloss PD}, \textbf{Westcott SL}. 2011. Assessing and
improving methods used in operational taxonomic unit-based approaches
for 16S rRNA gene sequence analysis. Applied and Environmental
Microbiology \textbf{77}:3219--3226.
doi:\href{https://doi.org/10.1128/aem.02810-10}{10.1128/aem.02810-10}.

\hypertarget{ref-Westcott2015}{}
2. \textbf{Westcott SL}, \textbf{Schloss PD}. 2015. De novo clustering
methods outperform reference-based methods for assigning 16S rRNA gene
sequences to operational taxonomic units. PeerJ \textbf{3}:e1487.
doi:\href{https://doi.org/10.7717/peerj.1487}{10.7717/peerj.1487}.

\hypertarget{ref-He2015}{}
3. \textbf{He Y}, \textbf{Caporaso JG}, \textbf{Jiang X-T},
\textbf{Sheng H-F}, \textbf{Huse SM}, \textbf{Rideout JR}, \textbf{Edgar
RC}, \textbf{Kopylova E}, \textbf{Walters WA}, \textbf{Knight R},
\textbf{Zhou H-W}. 2015. Stability of operational taxonomic units: An
important but neglected property for analyzing microbial diversity.
Microbiome \textbf{3}.
doi:\href{https://doi.org/10.1186/s40168-015-0081-x}{10.1186/s40168-015-0081-x}.

\hypertarget{ref-Kopylova2016}{}
4. \textbf{Kopylova E}, \textbf{Navas-Molina JA}, \textbf{Mercier C},
\textbf{Xu ZZ}, \textbf{Mahé F}, \textbf{He Y}, \textbf{Zhou H-W},
\textbf{Rognes T}, \textbf{Caporaso JG}, \textbf{Knight R}. 2016.
Open-source sequence clustering methods improve the state of the art.
mSystems \textbf{1}:e00003--15.
doi:\href{https://doi.org/10.1128/msystems.00003-15}{10.1128/msystems.00003-15}.

\hypertarget{ref-Edgar2013}{}
5. \textbf{Edgar RC}. 2013. UPARSE: Highly accurate OTU sequences from
microbial amplicon reads. Nature Methods \textbf{10}:996--998.
doi:\href{https://doi.org/10.1038/nmeth.2604}{10.1038/nmeth.2604}.

\hypertarget{ref-Mah2014}{}
6. \textbf{Mahé F}, \textbf{Rognes T}, \textbf{Quince C}, \textbf{Vargas
C de}, \textbf{Dunthorn M}. 2014. Swarm: Robust and fast clustering
method for amplicon-based studies. PeerJ \textbf{2}:e593.
doi:\href{https://doi.org/10.7717/peerj.593}{10.7717/peerj.593}.

\hypertarget{ref-Mah2015}{}
7. \textbf{Mahé F}, \textbf{Rognes T}, \textbf{Quince C}, \textbf{Vargas
C de}, \textbf{Dunthorn M}. 2015. Swarm v2: Highly-scalable and
high-resolution amplicon clustering. PeerJ \textbf{3}:e1420.
doi:\href{https://doi.org/10.7717/peerj.1420}{10.7717/peerj.1420}.

\hypertarget{ref-Barriuso2011}{}
8. \textbf{Barriuso J}, \textbf{Valverde JR}, \textbf{Mellado RP}. 2011.
Estimation of bacterial diversity using next generation sequencing of
16S rDNA: A comparison of different workflows. BMC Bioinformatics
\textbf{12}:473.
doi:\href{https://doi.org/10.1186/1471-2105-12-473}{10.1186/1471-2105-12-473}.

\hypertarget{ref-Bonder2012}{}
9. \textbf{Bonder MJ}, \textbf{Abeln S}, \textbf{Zaura E},
\textbf{Brandt BW}. 2012. Comparing clustering and pre-processing in
taxonomy analysis. Bioinformatics \textbf{28}:2891--2897.
doi:\href{https://doi.org/10.1093/bioinformatics/bts552}{10.1093/bioinformatics/bts552}.

\hypertarget{ref-Chen2013}{}
10. \textbf{Chen W}, \textbf{Zhang CK}, \textbf{Cheng Y}, \textbf{Zhang
S}, \textbf{Zhao H}. 2013. A comparison of methods for clustering 16S
rRNA sequences into OTUs. PLoS ONE \textbf{8}:e70837.
doi:\href{https://doi.org/10.1371/journal.pone.0070837}{10.1371/journal.pone.0070837}.

\hypertarget{ref-Huse2010}{}
11. \textbf{Huse SM}, \textbf{Welch DM}, \textbf{Morrison HG},
\textbf{Sogin ML}. 2010. Ironing out the wrinkles in the rare biosphere
through improved OTU clustering. Environmental Microbiology
\textbf{12}:1889--1898.
doi:\href{https://doi.org/10.1111/j.1462-2920.2010.02193.x}{10.1111/j.1462-2920.2010.02193.x}.

\hypertarget{ref-May2014}{}
12. \textbf{May A}, \textbf{Abeln S}, \textbf{Crielaard W},
\textbf{Heringa J}, \textbf{Brandt BW}. 2014. Unraveling the outcome of
16S rDNA-based taxonomy analysis through mock data and simulations.
Bioinformatics \textbf{30}:1530--1538.
doi:\href{https://doi.org/10.1093/bioinformatics/btu085}{10.1093/bioinformatics/btu085}.

\hypertarget{ref-Cai2011}{}
13. \textbf{Cai Y}, \textbf{Sun Y}. 2011. ESPRIT-tree: Hierarchical
clustering analysis of millions of 16S rRNA pyrosequences in quasilinear
computational time. Nucleic Acids Research \textbf{39}:e95--e95.
doi:\href{https://doi.org/10.1093/nar/gkr349}{10.1093/nar/gkr349}.

\hypertarget{ref-Sun2011}{}
14. \textbf{Sun Y}, \textbf{Cai Y}, \textbf{Huse SM}, \textbf{Knight R},
\textbf{Farmerie WG}, \textbf{Wang X}, \textbf{Mai V}. 2011. A
large-scale benchmark study of existing algorithms for
taxonomy-independent microbial community analysis. Briefings in
Bioinformatics \textbf{13}:107--121.
doi:\href{https://doi.org/10.1093/bib/bbr009}{10.1093/bib/bbr009}.

\hypertarget{ref-White2010}{}
15. \textbf{White JR}, \textbf{Navlakha S}, \textbf{Nagarajan N},
\textbf{Ghodsi M-R}, \textbf{Kingsford C}, \textbf{Pop M}. 2010.
Alignment and clustering of phylogenetic markers - implications for
microbial diversity studies. BMC Bioinformatics \textbf{11}:152.
doi:\href{https://doi.org/10.1186/1471-2105-11-152}{10.1186/1471-2105-11-152}.

\hypertarget{ref-AlGhalith2016}{}
16. \textbf{Al-Ghalith GA}, \textbf{Montassier E}, \textbf{Ward HN},
\textbf{Knights D}. 2016. NINJA-OPS: Fast accurate marker gene alignment
using concatenated ribosomes. PLOS Computational Biology
\textbf{12}:e1004658.
doi:\href{https://doi.org/10.1371/journal.pcbi.1004658}{10.1371/journal.pcbi.1004658}.

\hypertarget{ref-Sun2009}{}
17. \textbf{Sun Y}, \textbf{Cai Y}, \textbf{Liu L}, \textbf{Yu F},
\textbf{Farrell ML}, \textbf{McKendree W}, \textbf{Farmerie W}. 2009.
ESPRIT: Estimating species richness using large collections of 16S rRNA
pyrosequences. Nucleic Acids Research \textbf{37}:e76--e76.
doi:\href{https://doi.org/10.1093/nar/gkp285}{10.1093/nar/gkp285}.

\hypertarget{ref-Rideout2014}{}
18. \textbf{Rideout JR}, \textbf{He Y}, \textbf{Navas-Molina JA},
\textbf{Walters WA}, \textbf{Ursell LK}, \textbf{Gibbons SM},
\textbf{Chase J}, \textbf{McDonald D}, \textbf{Gonzalez A},
\textbf{Robbins-Pianka A}, \textbf{Clemente JC}, \textbf{Gilbert JA},
\textbf{Huse SM}, \textbf{Zhou H-W}, \textbf{Knight R}, \textbf{Caporaso
JG}. 2014. Subsampled open-reference clustering creates consistent,
comprehensive OTU definitions and scales to billions of sequences. PeerJ
\textbf{2}:e545.
doi:\href{https://doi.org/10.7717/peerj.545}{10.7717/peerj.545}.

\hypertarget{ref-Schmidt2014Limits}{}
19. \textbf{Schmidt TSB}, \textbf{Rodrigues JFM}, \textbf{Mering C von}.
2014. Limits to robustness and reproducibility in the demarcation of
operational taxonomic units. Environ Microbiol \textbf{17}:1689--1706.
doi:\href{https://doi.org/10.1111/1462-2920.12610}{10.1111/1462-2920.12610}.

\hypertarget{ref-Matthews1975}{}
20. \textbf{Matthews B}. 1975. Comparison of the predicted and observed
secondary structure of t4 phage lysozyme. Biochimica et Biophysica Acta
(BBA) - Protein Structure \textbf{405}:442--451.
doi:\href{https://doi.org/10.1016/0005-2795(75)90109-9}{10.1016/0005-2795(75)90109-9}.

\hypertarget{ref-Roesch2007}{}
21. \textbf{Roesch LFW}, \textbf{Fulthorpe RR}, \textbf{Riva A},
\textbf{Casella G}, \textbf{Hadwin AKM}, \textbf{Kent AD},
\textbf{Daroub SH}, \textbf{Camargo FAO}, \textbf{Farmerie WG},
\textbf{Triplett EW}. 2007. Pyrosequencing enumerates and contrasts soil
microbial diversity. The ISME Journal.
doi:\href{https://doi.org/10.1038/ismej.2007.53}{10.1038/ismej.2007.53}.

\hypertarget{ref-Schloss2012Stabilization}{}
22. \textbf{Schloss PD}, \textbf{Schubert AM}, \textbf{Zackular JP},
\textbf{Iverson KD}, \textbf{Young VB}, \textbf{Petrosino JF}. 2012.
Stabilization of the murine gut microbiome following weaning. Gut
Microbes \textbf{3}:383--393.
doi:\href{https://doi.org/10.4161/gmic.21008}{10.4161/gmic.21008}.

\hypertarget{ref-Pruesse2007}{}
23. \textbf{Pruesse E}, \textbf{Quast C}, \textbf{Knittel K},
\textbf{Fuchs BM}, \textbf{Ludwig W}, \textbf{Peplies J},
\textbf{Glockner FO}. 2007. SILVA: A comprehensive online resource for
quality checked and aligned ribosomal RNA sequence data compatible with
ARB. Nucleic Acids Research \textbf{35}:7188--7196.
doi:\href{https://doi.org/10.1093/nar/gkm864}{10.1093/nar/gkm864}.

\hypertarget{ref-Schloss2009}{}
24. \textbf{Schloss PD}, \textbf{Westcott SL}, \textbf{Ryabin T},
\textbf{Hall JR}, \textbf{Hartmann M}, \textbf{Hollister EB},
\textbf{Lesniewski RA}, \textbf{Oakley BB}, \textbf{Parks DH},
\textbf{Robinson CJ}, \textbf{Sahl JW}, \textbf{Stres B},
\textbf{Thallinger GG}, \textbf{Horn DJV}, \textbf{Weber CF}. 2009.
Introducing mothur: Open-source, platform-independent,
community-supported software for describing and comparing microbial
communities. Applied and Environmental Microbiology
\textbf{75}:7537--7541.
doi:\href{https://doi.org/10.1128/aem.01541-09}{10.1128/aem.01541-09}.

\hypertarget{ref-Rognes2015}{}
25. \textbf{Rognes T}, \textbf{Mahé F}, \textbf{Flouri T},
\textbf{McDonald; D}. 2015. Vsearch: VSEARCH 1.4.0. Zenodo.

\hypertarget{ref-Albanese2015}{}
26. \textbf{Albanese D}, \textbf{Fontana P}, \textbf{Filippo CD},
\textbf{Cavalieri D}, \textbf{Donati C}. 2015. MICCA: A complete and
accurate software for taxonomic profiling of metagenomic data. Sci Rep
\textbf{5}:9743.
doi:\href{https://doi.org/10.1038/srep09743}{10.1038/srep09743}.

\hypertarget{ref-Kopylova2012}{}
27. \textbf{Kopylova E}, \textbf{Noe L}, \textbf{Touzet H}. 2012.
SortMeRNA: Fast and accurate filtering of ribosomal RNAs in
metatranscriptomic data. Bioinformatics \textbf{28}:3211--3217.
doi:\href{https://doi.org/10.1093/bioinformatics/bts611}{10.1093/bioinformatics/bts611}.

\hypertarget{ref-Caporaso2010}{}
28. \textbf{Caporaso JG}, \textbf{Kuczynski J}, \textbf{Stombaugh J},
\textbf{Bittinger K}, \textbf{Bushman FD}, \textbf{Costello EK},
\textbf{Fierer N}, \textbf{Peña AG}, \textbf{Goodrich JK},
\textbf{Gordon JI}, \textbf{Huttley GA}, \textbf{Kelley ST},
\textbf{Knights D}, \textbf{Koenig JE}, \textbf{Ley RE},
\textbf{Lozupone CA}, \textbf{McDonald D}, \textbf{Muegge BD},
\textbf{Pirrung M}, \textbf{Reeder J}, \textbf{Sevinsky JR},
\textbf{Turnbaugh PJ}, \textbf{Walters WA}, \textbf{Widmann J},
\textbf{Yatsunenko T}, \textbf{Zaneveld J}, \textbf{Knight R}. 2010.
QIIME allows analysis of high-throughput community sequencing data.
Nature Methods \textbf{7}:335--336.
doi:\href{https://doi.org/10.1038/nmeth.f.303}{10.1038/nmeth.f.303}.

\end{document}
